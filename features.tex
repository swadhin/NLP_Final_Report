\section{Features}
\label{sec:features}
We will use three different groups of features to help improve sarcasm detection accuracy, including lexical features, syntactic features, and social features. Table \ref{tab:lexical features} shows the lexical features we used in our experiments. Lexical features mainly capture lexical patterns of tweets, such as common word usage, phrases, etc. Syntactic features illustrated in Table \ref{tab:syn features} represent syntactic structures of tweets, such as the simple SVO pattern, etc. We exploit social features listed in Table \ref{tab:social features} to characterize dependencies across tweets. For example, tweets with similar named entities tend to have same sentiments.

\begin{table}[htpb]
\centering
\begin{tabular}{|l|}
\hline
\tabincell{l}{Punctuation Frequencies\\
(The ratio between the number of punctuation \\
and the total number of tokens)} \\
\hline
Stopwords and Slang  \\
\hline
Emoticon Frequencies \\
\hline
Capitalization \\
\hline
Number of Hashtags and URLs \\
\hline
\tabincell{l}{Length of Sentence \\
(The number of tokens)}\\
\hline
Language Models(Unigram, Bigram and Trigram) \\              
\hline
\end{tabular}
\vspace{0.03 in}
\caption{Lexical Features}
\label{tab:lexical features}
\end{table}

\begin{table}
\centering
\begin{tabular}{|l|}
\hline
\tabincell{l}{Relative frequencies of different POS tags\\
parsed by the tool TweetNLP\cite{tweetnlp}\\
(The size of the tagset is $25$)}\\
\hline
\tabincell{l}{Fraction of words having syntactic function\\
(Syntactic dependency trees are generated by \\
TweeboParser\cite{kong2014dependency},some tokens do not \\
have syntactic function, such as ``RT'', ``@'', hashtags, etc.)} \\
\hline
\tabincell{l}{Number of inner nodes in dependency trees\\
(Phrases in a tweet can be characterized by subtrees,\\
thus the number of inner nodes is a reliable indicator\\
of the syntactic complexity of a tweet)}\\
\hline
\tabincell{l}{Sentiment of tweets\\
(The sentiment of a tweet can be positive,\\
negative or neutral. The online machine learning\\
framework Datumbox\cite{datumbox} is used to analyze sentiment.)}\\
\hline
\end{tabular}
\vspace{0.03 in}
\caption{Syntactic Features.}
\label{tab:syn features}
\end{table}

\begin{table}[htpb]
\centering
\begin{tabular}{|l|}
\hline
\tabincell{l}{Named Entities(The number of Named Entities,\\
the length of each named entity, the sentiment of each\\
named entity)}\\
\hline
\tabincell{l}{Social strength(The number of favorites,\\
the number of retweets, the follower count of the tweet\\
handler, the followee count of the tweet handler)}\\
\hline
\end{tabular}
\vspace{0.03 in}
\caption{Social Features}
\label{tab:social features}
\end{table}
