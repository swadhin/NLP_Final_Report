\section{Introduction}
\textbf{Motivation :} In recent years, social media sites such as Twitter have gained immense popularity and importance. These sites have
evolved into large platforms where users express their ideas and opinions freely. Companies leverage this unique
ecosystem to tap into public opinion on their products or services and to provide real-time customer assistance. With the high velocity and volume of social media data, companies rely on tools such as HootSuite\footnote{https://hootsuite.com/}, to analyze data and to provide customer service. These tools perform tasks such as content management, sentiment analysis, and extraction of relevant messages for the company's customer service representatives to respond to. However, these tools lack the sophistication to decipher more nuanced forms of language such as sarcasm or humor, in which the meaning of a message is not always obvious and explicit.\\

Our goal in this project is to tackle the difficult problem of sarcasm detection on Twitter. While sarcasm detection
is inherently challenging, the style and nature of content on Twitter further complicate the process. Compared to other,
more conventional sources such as news articles and novels, Twitter is (1) more informal in nature with an evolving vocabulary of slang words and abbreviations and (2) has a limit of $140 $characters per tweet which provides fewer word level
cues and adds more ambiguity. However, it also different side channels to get some information about the context of a particular tweet like different hashtags, mentions, profile information, general sentiment of the named entities in tweets etc. To put the problem in a perspective, a few examples of sarcastic tweets from our experimental datasets are given below:
\begin{enumerate}
 \item \textit{Oh wow broken up 4 days and you've moved on already, thanks, don't feel like shit at all. \#sarcasm}.\\
 \item \textit{No my roommate play out of tune Zeppelin songs right outside my door isnt annoying. Not at all \#sarcasm \#sigh}.\\
 \item \textit{Wow! @TWCable\_NYC Thanks for the option of high speed internet at \$5 a month or 6 months free to save \$$0.30$ depending on the plan. \#sarcasm}.\\
 \item \textit{Man, Robinson Cano could be the laziest MVP ever \#sarcasm \#bigtimesarcasm}.\\
 \item \textit{20 minutes of laundry at 1 am. Awesome \#sarcasm}.\\
 \item \textit{Love walking through your cloud of cigarette smoke. Why buy my own pack when I can just inhale yours \#sarcasm}.\\
 \item \textit{Is everyone as excited about the \#GOP and \#Democrat conventions as I am?!? \#sarcasm \#serioussarcasm \#deepyawningmawofsarcaam}.\\
\end{enumerate}

\textbf{Our Approach :} Current research on sarcasm detection on Twitter \cite{riloff13,davidov10,tomas14,gonzalez_acl} has primarily focused on obtaining information from the text of the tweets. These techniques treat sarcasm as a linguistic phenomenon, with limited emphasis on the socio-contextual aspects of sarcasm. However, sarcasm has been extensively studied in psychological and behavioral sciences and theories explaining when, why, and how sarcasm is expressed have been established. All these works point that sarcasm is bound with broader common knowledge (e.g., about news or celebrities), the context known only to the author or author's opinion.\\

Hence, to follow a systematic approach, we first use and extend different lexical and syntactic features used in \cite{riloff13,davidov10,tomas14,gonzalez_acl} to capture the literal form of sarcasm. We explicitly utilize the occurrence of punctuation, stop words, slang, and emoticons which are very closely related to sentiment representation of tweets. Although twitter users think independently when updating their statuses, certain level of similarities in lexical structures are expected. Secondly, compared with normal tweets, more complicated syntactic structures may appear in sarcastic tweets so that implicit meaning of them can be understood by others. Syntactic complexity is an indicator of how easily the information of a tweet can be understood. Conventionally, sarcastic tweets convey messages implicitly, which possibly complicates syntactic structures. Finally and most importantly, we get different socio-contextual information from the information follower, followee, retweets, mentions etc. which represent the social context of the user and the tweet. So, we combine these features to train a supervised learning algorithm to detect sarcasm. We have collected sarcastic tweets with corresponding socio-contextual information using \#sarcasm hashtag and also later consolidated the dataset upto 75,253 tweets consisting 38,112 sarcastic tweets by getting tweets from the publicly available tweet ids from the authors of {riloff13,davidov10,tomas14}. We used SVM and ensemble technique like Logitboost or bagging to get the best F1-score of 0.9746 using this feature set in the above dataset.\\ 

\noindent We make the following contributions in this project:
\begin{itemize}
 \item We have created a Twitter dataset of reasonable amount of tweets with relevant socio-contextual information.
 \item To the best of our knowledge, we have employed sophisticated syntactic and social features to capture the context to better detect sarcasm. This approach can help in building language independent sarcasm detector.\\
\end{itemize}

In Sec. \ref{sec:related}, we review related sarcasm detection research. In Sec. \ref{sec:problem}, we formally define sarcasm detection on Twitter. In Sec. \ref{sec:dataset}, we discuss dataset collection procedure in detail and then we discuss the nature of different sarcastic tweets. In Sec. \ref{sec:features} we discuss different feature set and in Sec. \ref{sec:methodology} we describe the methodology employed in sarcasm detection. Next, in Sec. \ref{sec:evaluation}, we present experimental setup and different evaluation results. Finally, in Sec. \ref{sec:future} and Sec. \ref{sec:conclusion}, we conclude with different possible future directions.



