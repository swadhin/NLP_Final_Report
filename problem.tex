\section{Problem Definition}
\label{sec:problem}
Sarcasm, while similar to irony, differs in that it is usually viewed as being caustic and derisive. Some researchers even
consider it to be aggressive humor and a form of verbal aggression. While researchers in linguistics and psychology debate about what exactly constitutes sarcasm, for the
sake of clarity, we use \textit{the activity of saying or writing the opposite of what you mean, or of speaking in a way intended to make someone
else feel stupid or show them that you are angry} (Macmillan English Dictionary (2007)). We formally define the sarcasm detection problem on Twitter as follows:
\begin{mydef}
\textbf{Sarcasm Detection on Twitter.} Given an
unlabeled tweet t from user U along with a set of U’s and t's social information S, a solution to sarcasm detection aims to automatically detect if t is sarcastic or not.
\end{mydef}
So, our problem is different from past sarcasm detection research
which use only text information from t and do not consider
the user's and tweet's social information S that are available on Twitter. In SSD, we train a classifier system with a training tweet dataset using different features and test the prediction accuracy in the test dataset.