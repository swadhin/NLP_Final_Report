\section{Dataset Collection and Description}
\label{sec:dataset}
\subsection{Crawler Development}
To collect a large-scale dataset from Twitter, we developed a distributed crawling platform to parallelize the crawling process. The open-source tool Tweepy \cite{tweepy} was utilized to use APIs from Twitter application development framework. Twitter has limited each account to send $150$ requests within a $15$-min time window. To speed up the crawling process, we created $7$ twitter accounts to send requests in parallel. Sarcastic tweet dataset consists of the recent $10k$ tweets containing hashtags \emph{\#sarcasm}. We collected comprehensible information for each tweet, including the tweeter, tweet text, post time, the count of favorites, the count of retweets, etc. From these tweets, we further collect information for each tweeter. For each Twitter, we collected its follower count, followee count, user ID, status count, the list of followers, the list of followees, etc. Moreover, we have also collected non-sarcastic tweets of around $10k$ from different accounts.\\

Furthermore, we have contacted different authors \cite{davidov10,riloff13,tomas14} for similar works for twitter dataset, but they were unable to provide contents of tweets due to new Twitter terms of services \cite{twitter_tos}. Instead, authors \cite{davidov10,riloff13,tomas14} have provided us twitter ids corresponding with their labeling (These labels are done either by Amazon Turkers or by independent human evaluators). Then, we used our crawlers to collect tweets from those tweet ids with their socio-contextual information (on an average $10\%$ of the tweets were either deleted or inaccessible). Finally, we are able to collect 180 tweets annotated from \cite{davidov10}, around 50k tweets from \cite{tomas14} and around 5k tweets from \cite{riloff13}. Thus, we consolidated twitter dataset of size $75,213$ tweets consisting $38,112$ sarcastic tweets.

\subsection{Dataset Description}
The whole dataset is divided into three parts: tweets information, user profiles and user social relationships.

Table \ref{tab:tweet information} shows the format of a tweet information record. Note that, we only list the information we have used in our experiments. A tweet record in our dataset also includes other information, such as user device, geo-location, etc. 

A user profile record consists of the user name, followers count, followees count and status count. Table \ref{tab:user profile} illustrates the format for a user profile record. Details of the user social relationship dataset can be seen in Table \ref{tab:social relationship}.

\begin{table}[htpb]
\centering
\begin{tabular}{|c|c|c|c|}
\hline
Tweeter  & Text  & Favoriates Count & Retweets Count \\
\hline
\end{tabular}
\vspace{0.01 in}
\caption{Format for tweet information dataset.}
\label{tab:tweet information}
\end{table}

\begin{table}[htpb]
\centering
\begin{tabular}{|c|c|c|c|}
\hline
User Name & Followers Count & Followees Count & Status Count \\
\hline
\end{tabular}
\vspace{0.01 in}
\caption{Format for user profile dataset.}
\label{tab:user profile}
\end{table}

\begin{table}[htpb]
\centering
\begin{tabular}{|c|c|c|}
\hline
User Name & Follower1, Follower2, $\cdots$ & Followee1, Followee2, $\cdots$ \\
\hline
\end{tabular}
\vspace{0.01 in}
\caption{Format for user social relationship dataset.}
\label{tab:social relationship}
\end{table}