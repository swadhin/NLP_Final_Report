\section{Future Works}
\label{sec:future}
In this project, our goal was to capture the context of tweets in a holistic manner. To achieve this, one immediate future direction might be to get previous tweets of a user to get a better behavioral model which might help in prediction. We can also better model the persona or community of a twitter handle if we also can mine the interest from his follower list and previous retweets/mentions/favorites. Similarly, we can also get previous tweets belonging to particular topic or named entity mentioned in the tweet and get to know the general sentiment. If the general sentiment is negative, then any positive tweet, with high probability, might be sarcastic. Furthermore, in future, we will also attempt to get information about the topics or entities from outside sources such as search APIs. On the other hand, we can develop a hierarchical classifier for different types of sarcastic tweets which will initially predict the class of sarcastic tweet (e.g. general or entity specific) and then employ a class specific model to predict the sarcasm. We believe that this type of topical or socio-contextual model will help in predicting sarcastic tweets in a language independent way and can overcome the low F1-score reported in Czech dataset in \cite{tomas14}. Moreover, as \#sarcasm hashtag is a bit noisy, in future, we will try to employ amazon mechanical turkers for more confidence in judgment. We also should have investigated the importance of different features in terms of information gain.