\section{Discussion on Sarcastic Tweets}
\label{sec:dataset}

Based on our observations, sarcastic tweets reveal both independent and dependent features. Generally, independent features characterize lexical or syntactic structures of tweets no matter who posted them and what the content or topic of them is. While social network specific characteristics can reveal dependencies among tweets to some extent.\\

Lexical structures indicate similarities of using words or characters in raw text of tweets. The frequencies of punctuation, stopwords and slang represent one aspect of lexical structures. We explicitly utilize the occurrence of emoticons which are very closely related to sentiment representation of tweets. Moreover, hashtags and URLs are special lexicons in tweets, and they are possibly used to clarify the information conveyed in a sarcastic tweet. Some common phrases may be used to reverse the polarity of an utterance. Language models (such as Bigram, Trigram) are efficient ways to find out these phrases. Although twitter users think independently when updating their statues, certain level of similarities in lexical structures are expected to see due to their common intentions to represent sarcasms to their friends.\\

Compared with normal tweets, more complicated syntactic structures may appear in sarcastic tweets so that implicit meaning of them can be understood by others. The complexity of a syntactic tree can be represented by the number of nodes in the tree, the fraction of words appeared in the syntactic tree and the total words in tweet texts, etc. Syntactic complexity is an indicator of how easily the information of a tweet can be understood. Conventionally, sarcastic tweets convey messages implicitly, which possibly complicates syntactic structures.\\

Messages in social networks somewhat show some dependencies. On one hand, when people talk about a specific entity (such as a person, a movie, etc.), they tend to describe their opinions in similar ways. And we can always see common sentiments for some entities. For example, when someone wants to comment a movie, posting a sarcastic message can possibly invoke others' interests in following that message.\\

Named entities represent the target or topic of a tweet. After recognizing named entities appeared in tweets, we are able to classify tweets into different types. For example, tweets about general events (such as weather) and tweets targeting at specific entities (like movies, actors, etc). Moreover, common sentiments towards specific entities can also possibly improve the accuracy of classification. The sentiment of the whole tweet is considered as the sentiment of named entities in that tweet. The number of named entities in a tweet also represents the intention that a user wants to convey complex messages. On the other hand, the social strength of a tweet can be defined as the number of favorites, retweets, the number of followers and followees the poster has. Generally, social strength can be considered as a factor which measures others' interests in a tweet. Posting sarcastic tweets is an efficient way to attract others' attention. Thus, tweets having higher social strength tend to be more likely sarcastic tweets. Exploiting dependencies can cluster tweets into several types having common properties.\\

In Table \ref{tab:sarcastic tweets}, we can see several representative sarcastic tweets which show dependencies we are going to exploit. All these tweets contain a common verb ``love'' which indicates positive context. Detecting contrary sentiment between the context and the target of the tweet for tweets with simple sentence structure is hard. For example, the target ``growing up'' is generally a negative thing. Language model is an efficient way to identify this kind of tweets since they only contain very simple phrases. We need more features such as syntactic features to classify general sarcastic tweets having complicated sentence structures. As for sarcastic tweets with named entities, they generally tend to have similar sentiment.

\begin{table}[htpb]
\centering
\begin{tabular}{|c|c|}
\hline
Sarcastic Tweet Types & Example \\
\hline
\tabincell{c}{General Sarcastic Tweets \\
with simple sentence structure} & 

\tabincell{c}{I love growing up \#sarcasm} \\
\hline
\tabincell{c}{General Sarcastic Tweets\\
with complicated sentence structure} & 

\tabincell{c}{I love when I can't find my \\
                                        resources for my lessons that I \\
                                        thought I had on my desk. \\
                                         \#yeg \#ualberta \#sarcasm} \\
 \hline
\tabincell{c}{Sarcastic Tweets \\
 with Named Entities} &

\tabincell{c}{I just love Kalpana Bales\\
              voice and the way she sings.\\
              \#sarcastic}\\
\hline
\end{tabular}
\vspace{0.03 in}
\caption{Representative sarcastic tweets showing dependencies.}
\label{tab:sarcastic tweets}
\end{table}
