\begin{abstract}
Sarcasm is a nuanced form of language in which individuals state the opposite of what is implied. With this intentional
ambiguity, sarcasm detection has always been a challenging task, even for humans. Recognition of sarcasm can benefit many sentiment analysis NLP applications, such as review summarization, dialogue systems and review ranking systems. Moreover, this problem of sarcasm detection becomes very important in this age of mining public opininion from social media such as Twitter. Current approaches for automatic sarcasm detection in Twitter rely primarily on lexical cues of an independent tweets. However, the essence of sarcasm depends upon the context and world knowledge. This project aims to address the difficult task of sarcasm detection on Twitter by leveraging social features such as follower list or profile description, contextual feature such as sentiment expressed in the named entities in tweets etc. intrinsic to users expressing sarcasm with lexical and linguistic features. We evaluate our technique, \textit{Supervised Sarcasm Detection} (SSD) on a dataset of 75,253 tweets consisting 38,112 sarcastic tweets having \textit{$\#sarcasm$} hashtag. We demonstrate efficiency of \textit{SSD} in identifying sarcastic tweets by gaining F1-score of 0.9746 which outperforms different state-of-the-art techniques.
\end{abstract}
